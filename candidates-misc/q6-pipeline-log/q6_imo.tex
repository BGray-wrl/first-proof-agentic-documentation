\documentclass[11pt]{article}

% --- Page Layout ---
\usepackage{geometry}
\geometry{margin=1.25in}

% --- Math Packages ---
\usepackage{amsmath, amssymb, amsthm}

% --- Colored Boxes ---
\usepackage[most]{tcolorbox}
\tcbuselibrary{skins, breakable}

% --- Box Colors ---
\definecolor{boxblue}{RGB}{0,0,150}
\definecolor{boxback}{RGB}{245,245,255}

% --- Theorem Environments ---
\theoremstyle{plain}
\newtheorem{theorem}{Theorem}
\newtheorem{lemma}{Lemma}
\newtheorem{proposition}{Proposition}

\theoremstyle{definition}
\newtheorem{definition}{Definition}
\newtheorem{remark}{Remark}

% --- Problem Box ---
\newtcolorbox{problem}[1]{%
  colback=boxback,
  colframe=boxblue,
  fonttitle=\bfseries\large,
  title={#1},
  sharp corners,
  enhanced,
  attach boxed title to top left={yshift=-2mm, xshift=2mm},
  boxed title style={colframe=boxblue, colback=boxblue},
  before skip=15pt plus 2pt,
  after skip=15pt plus 2pt,
  top=10pt, bottom=10pt, left=10pt, right=10pt
}

% --- Result Box (breakable across pages) ---
\newtcolorbox{result}[1]{%
  colback=white,
  colframe=boxblue,
  fonttitle=\bfseries\large,
  title={#1},
  sharp corners,
  enhanced jigsaw,
  breakable,
  attach boxed title to top left={yshift=-2mm, xshift=2mm},
  boxed title style={colframe=boxblue, colback=boxblue},
  before skip=15pt plus 2pt,
  after skip=15pt plus 2pt,
  top=10pt, bottom=10pt, left=10pt, right=10pt
}

% Manually added for debugging
\newcommand{\e}{\mathrm e}
\newcommand{\1}{\mathbf{1}}

\begin{document}

\begin{problem}{User Prompt}
For a graph $G = (V, E)$, let $G_S = (V, E(S,S))$ denote the graph with the same vertex set, 
but only the edges between vertices in $S$. Let $L$ be the Laplacian matrix of $G$ and let $L_S$ be the Laplacian of $G_S$. 
I say that a set of vertices $S$ is $\epsilon$-light if the matrix $\epsilon L - L_S$ is positive semidefinite. 
Does there exist a constant $c > 0$ so that for every graph $G$ and every $\epsilon$ between $0$ and $1$, $V$ contains an $\epsilon$-light subset $S$ of size at least $c \epsilon |V|$? 

\end{problem}

\begin{result}{System Result}

\begin{theorem}[What the ``heavy-edge'' approach proves, and where it stops]\label{thm:gap}
Let $G=(V,E)$ be a finite, connected, simple graph on $n:=|V|$ vertices, with (combinatorial) Laplacian
\[
L \;=\; \sum_{\{u,v\}\in E} b_{\{u,v\}}b_{\{u,v\}}^{\top}, \qquad b_{\{u,v\}}:=\mathbf e_u-\mathbf e_v.
\]
For $S\subseteq V$, let $L_S$ be the Laplacian of the induced subgraph $G[S]$, embedded as an $n\times n$ matrix.
Fix $\varepsilon\in(0,1)$.

\smallskip
\noindent
{\bf (A) (Combinatorial reduction.)}
There exists $S\subseteq V$ with $|S|\ge \frac{\varepsilon n}{8\e}$ such that $G[S]$ contains no edge $e$ whose effective resistance
$R_{\mathrm{eff}}(e)$ (in $G$) exceeds $\varepsilon$.

\smallskip
\noindent
{\bf (B) (The missing spectral step.)}
The conclusion in (A) \emph{does not} imply that $S$ is $\varepsilon$-light (i.e.\ that $L_S\preceq \varepsilon L$). In particular,
even if every edge of $G[S]$ satisfies $R_{\mathrm{eff}}(e)\le \varepsilon$, it may still happen that $L_S\not\preceq \varepsilon L$.

\smallskip
\noindent
{\bf (C) (An exact spectral reformulation.)}
For each $u\in V$ define the star Laplacian
\[
L_u \;:=\; \sum_{\substack{v:\ \{u,v\}\in E}} b_{\{u,v\}}b_{\{u,v\}}^{\top}.
\]
Then for every $S\subseteq V$ one has
\[
L_S \;\preceq\; \frac12\sum_{u\in S} L_u.
\]
Consequently, to prove that $S$ is $\varepsilon$-light it suffices to show
\[
\sum_{u\in S} L_u \;\preceq\; 2\varepsilon\,L.
\]
The candidate solution implicitly requires a ``log-free'' matrix-selection/rounding statement of this type; no such statement is proved there.
\end{theorem}

\begin{proof}
\textbf{Effective resistance preliminaries.}
Let $L^\dagger$ denote the Moore--Penrose pseudoinverse. For an edge $e=\{u,v\}$, define
\[
R_{\mathrm{eff}}(e)\;:=\; b_e^{\top}L^\dagger b_e.
\]
Foster's theorem states that
\[
\sum_{e\in E} R_{\mathrm{eff}}(e) \;=\; n-1,
\]
see Theorem~5.1 of Ge~\cite[Theorem~5.1]{GeFoster} (an arXiv preprint which quotes Foster's original result and provides a proof).

\smallskip
\noindent
\textbf{Proof of (A).}
Define the set of ``heavy'' edges
\[
E_{\mathrm{heavy}} \;:=\; \{e\in E:\ R_{\mathrm{eff}}(e)>\varepsilon\}.
\]
By Foster's theorem,
\[
|E_{\mathrm{heavy}}|\,\varepsilon
\;<\;
\sum_{e\in E_{\mathrm{heavy}}} R_{\mathrm{eff}}(e)
\;\le\;
\sum_{e\in E} R_{\mathrm{eff}}(e)
\;=\; n-1,
\]
so $|E_{\mathrm{heavy}}|< n/\varepsilon$.

Let $H=(V,E_{\mathrm{heavy}})$ and write $d_H(v)$ for the degree of $v$ in $H$. Then
\[
\sum_{v\in V} d_H(v) \;=\; 2|E_{\mathrm{heavy}}| \;<\; \frac{2n}{\varepsilon}.
\]
Let $V_{\mathrm{high}}:=\{v\in V:\ d_H(v)>\frac{4}{\varepsilon}\}$ and $V_{\mathrm{low}}:=V\setminus V_{\mathrm{high}}$.
By Markov's inequality,
\[
|V_{\mathrm{high}}|\cdot \frac{4}{\varepsilon} \;<\; \frac{2n}{\varepsilon}
\quad\Rightarrow\quad
|V_{\mathrm{high}}|<\frac n2
\quad\Rightarrow\quad
|V_{\mathrm{low}}|\ge \frac n2.
\]
Let $H_{\mathrm{low}}:=H[V_{\mathrm{low}}]$; then $\Delta(H_{\mathrm{low}})\le \lfloor 4/\varepsilon\rfloor$.

Now sample $S_{\mathrm{raw}}\subseteq V_{\mathrm{low}}$ by including each vertex independently with probability $p:=\varepsilon/4$.
To obtain an independent set in $H_{\mathrm{low}}$, fix an arbitrary total order on $V_{\mathrm{low}}$ and keep a vertex $v\in S_{\mathrm{raw}}$
iff no larger neighbor of $v$ (in $H_{\mathrm{low}}$) lies in $S_{\mathrm{raw}}$; call the resulting set $S$.
Then $S$ is an independent set in $H_{\mathrm{low}}$ (hence in $H$), so $G[S]$ contains no edge from $E_{\mathrm{heavy}}$.

Moreover, for each $v\in V_{\mathrm{low}}$,
\[
\mathbb P(v\in S)
\;\ge\;
p\,(1-p)^{d_{H_{\mathrm{low}}}(v)}
\;\ge\;
p\,(1-p)^{4/\varepsilon},
\]
and with $p=\varepsilon/4$ one has $(1-p)^{4/\varepsilon}\ge \e^{-1}$ (since $(1-x)^{1/x}\ge \e^{-1}$ for $x\in(0,1)$).
Therefore,
\[
\mathbb E|S|
\;=\;
\sum_{v\in V_{\mathrm{low}}}\mathbb P(v\in S)
\;\ge\;
|V_{\mathrm{low}}|\cdot \frac{\varepsilon}{4}\cdot \e^{-1}
\;\ge\;
\frac{n}{2}\cdot \frac{\varepsilon}{4}\cdot \e^{-1}
\;=\;
\frac{\varepsilon n}{8\e}.
\]
In particular, there exists a realization with $|S|\ge \varepsilon n/(8\e)$, and by construction $E(S)\cap E_{\mathrm{heavy}}=\varnothing$.
This completes (A).

\smallskip
\noindent
\textbf{Proof of (B).}
We first record a standard inequality relating effective resistance to quadratic forms.
For any edge $e=\{u,v\}$ and any $x\in\mathbb R^V$ with $x_u\neq x_v$, set $y:=x/(x_u-x_v)$ so that $y_u-y_v=1$. By the variational
characterization of effective resistance,
\[
R_{\mathrm{eff}}(e)
\;=\;
\min\{z^{\top}Lz:\ z_u-z_v=1\}
\;\le\;
y^{\top}Ly
\;=\;
\frac{x^{\top}Lx}{(x_u-x_v)^2}.
\]
Equivalently,
\begin{equation}\label{eq:edge-domination}
(x_u-x_v)^2 \;\le\; R_{\mathrm{eff}}(e)\, x^{\top}Lx
\qquad\text{for all }x\in\mathbb R^V,
\end{equation}
which is the matrix inequality $b_e b_e^{\top}\preceq R_{\mathrm{eff}}(e)\,L$.

Summing \eqref{eq:edge-domination} over $e\in E(S)$ gives the sufficient (but generally weak) bound
\[
x^{\top}L_S x
\;=\;
\sum_{e\in E(S)} (x_u-x_v)^2
\;\le\;
\Big(\sum_{e\in E(S)} R_{\mathrm{eff}}(e)\Big)\, x^{\top}Lx
\quad\Rightarrow\quad
L_S \;\preceq\; \Big(\sum_{e\in E(S)} R_{\mathrm{eff}}(e)\Big)\,L.
\]
Thus, controlling \emph{each} $R_{\mathrm{eff}}(e)$ by $\varepsilon$ does not force $L_S\preceq \varepsilon L$ unless one also
controls the \emph{total} sum of resistances over edges in $E(S)$, which can be large.

A concrete example is the complete graph $G=K_n$. For $K_n$, every edge has effective resistance
$R_{\mathrm{eff}}(e)=2/n$ (this is a standard computation; it also follows from symmetry and Foster's theorem),
so for any fixed $\varepsilon>0$ and all sufficiently large $n$ one has $R_{\mathrm{eff}}(e)\le \varepsilon$ for \emph{every} edge.
Nevertheless, if $S\subseteq V$ has size $|S|=s$, then (on $\1^\perp$) the Laplacian eigenvalues satisfy
$\lambda_{\max}(L)=n$ and $\lambda_{\max}(L_S)=s$ (since $L(K_m)$ has eigenvalue $m$ with multiplicity $m-1$),
so $L_S\preceq \varepsilon L$ would force $s\le \varepsilon n$. Taking, say, $s=\lfloor n/2\rfloor$ and $\varepsilon=0.1$
shows that ``all edges light'' does not imply $\varepsilon$-lightness of $S$. This proves (B).

\smallskip
\noindent
\textbf{Proof of (C).}
Expand $\sum_{u\in S}L_u$ as a sum over edges. Each induced edge $e=\{u,v\}\in E(S)$ appears
\emph{twice} in $\sum_{u\in S}L_u$ (once in $L_u$ and once in $L_v$), while each edge crossing
$S$ and $V\setminus S$ appears \emph{once}, and all other edges do not appear. Therefore,
\[
\frac12\sum_{u\in S}L_u
\;=\;
\sum_{e\in E(S)} b_e b_e^{\top}
\;+\;
\frac12\sum_{e\in \partial S} b_e b_e^{\top}
\;\succeq\;
\sum_{e\in E(S)} b_e b_e^{\top}
\;=\;
L_S,
\]
where $\partial S$ denotes edges with exactly one endpoint in $S$. Hence $L_S\preceq \frac12\sum_{u\in S}L_u$, and the stated
sufficient condition follows immediately. This completes (C).
\end{proof}

\begin{remark}[Why interlacing-polynomial citations do \emph{not} complete the candidate proof]
The candidate solution asserts that logarithmic losses from matrix concentration can be removed ``by interlacing polynomials''.
One precise interlacing-polynomial existence result is Theorem~1.4 of Marcus--Spielman--Srivastava~\cite[Theorem~1.4]{MSSKS}
(arXiv:1306.3969), which guarantees a norm bound for a sum $\sum_i v_i v_i^\ast$ of independent random rank-one matrices
under the hypotheses $\sum_i \mathbb E[v_i v_i^\ast]=I$ and $\mathbb E\|v_i\|^2\le \epsilon$.
However, applying \cite[Theorem~1.4]{MSSKS} would require a representation of the \emph{vertex-induced} Laplacian $L_S$
(or of $\sum_{u\in S}L_u$) as a sum of independent rank-one matrices whose individual contributions are uniformly small
in the relevant normalization. The candidate proof does not provide such a representation, nor does it establish
a matrix-selection/rounding principle of the form in Theorem~\ref{thm:gap}(C). Thus the interlacing-polynomial citation,
as used there, does not presently fill the gap.
\end{remark}

\begin{thebibliography}{9}

\bibitem{GeFoster}
J.~Ge,
\newblock \emph{Effective Resistances of Nearly Complete Bipartite Graphs},
\newblock arXiv:1904.07766.
\newblock (Foster's theorem stated as Theorem~5.1.)

\bibitem{TroppChernoff}
J.~A.~Tropp,
\newblock \emph{User-Friendly Tail Bounds for Sums of Random Matrices},
\newblock arXiv:1004.4389.
\newblock (Matrix Chernoff bounds stated as Theorem~1.1.)

\bibitem{MSSKS}
A.~W.~Marcus, D.~A.~Spielman, and N.~Srivastava,
\newblock \emph{Interlacing Families II: Mixed Characteristic Polynomials and the Kadison--Singer Problem},
\newblock arXiv:1306.3969.
\newblock (Interlacing existence bound stated as Theorem~1.4.)

\end{thebibliography}

\end{result}

\end{document}
