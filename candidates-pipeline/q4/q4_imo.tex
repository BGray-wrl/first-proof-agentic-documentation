\documentclass[11pt]{article}

% --- Page Layout ---
\usepackage{geometry}
\geometry{margin=1.25in}

% --- Math Packages ---
\usepackage{amsmath, amssymb, amsthm}

% --- Colored Boxes ---
\usepackage[most]{tcolorbox}
\tcbuselibrary{skins, breakable}

% --- Box Colors ---
\definecolor{boxblue}{RGB}{0,0,150}
\definecolor{boxback}{RGB}{245,245,255}

% --- Theorem Environments ---
\theoremstyle{plain}
\newtheorem{theorem}{Theorem}
\newtheorem{lemma}{Lemma}
\newtheorem{proposition}{Proposition}

\theoremstyle{definition}
\newtheorem{definition}{Definition}
\newtheorem{remark}{Remark}

% --- Problem Box ---
\newtcolorbox{problem}[1]{%
  colback=boxback,
  colframe=boxblue,
  fonttitle=\bfseries\large,
  title={#1},
  sharp corners,
  enhanced,
  attach boxed title to top left={yshift=-2mm, xshift=2mm},
  boxed title style={colframe=boxblue, colback=boxblue},
  before skip=15pt plus 2pt,
  after skip=15pt plus 2pt,
  top=10pt, bottom=10pt, left=10pt, right=10pt
}

% --- Result Box (breakable across pages) ---
\newtcolorbox{result}[1]{%
  colback=white,
  colframe=boxblue,
  fonttitle=\bfseries\large,
  title={#1},
  sharp corners,
  enhanced jigsaw,
  breakable,
  attach boxed title to top left={yshift=-2mm, xshift=2mm},
  boxed title style={colframe=boxblue, colback=boxblue},
  before skip=15pt plus 2pt,
  after skip=15pt plus 2pt,
  top=10pt, bottom=10pt, left=10pt, right=10pt
}

\begin{document}

\begin{problem}{User Prompt}
Let $p(x)$ and $q(x)$ be two monic polynomials of degree $n$:
\[
p(x) = \sum_{k=0}^n a_k x^{n-k} \quad \text{and} \quad q(x) = \sum_{k=0}^n
b_k x^{n-k}
\]
where $a_0 = b_0 = 1$. Define $p \boxplus_n q(x)$ to be the polynomial
\[
(p \boxplus_n q)(x) = \sum_{k=0}^n c_k x^{n-k}
\]
where the coefficients $c_k$ are given by the formula:
\[
c_k = \sum_{i+j=k} \frac{(n-i)! (n-j)!}{n! (n-k)!} a_i b_j
\]
for $k = 0, 1, \dots, n$.
For a monic polynomial $p(x)=\prod_{i\le n}(x- \lambda_i)$, define 
$$\Phi_n(p):=\sum_{i\le n}(\sum_{j\neq i} \frac1{\lambda_i-\lambda_j})^2$$ and $\Phi_n(p):=\infty$ if $p$ has a multiple root.
Is it true that if $p(x)$ and $q(x)$ are monic real-rooted polynomials of
degree $n$, then
$$\frac1{\Phi_n(p\boxplus_n q)} \ge \frac1{\Phi_n(p)}+\frac1{\Phi_n(q)}?$$

\end{problem}


\begin{result}{System Result:}

\begin{theorem}[Verified special cases; status of the general inequality]
Fix $n\ge 1$.  For monic polynomials
\[
p(x)=\sum_{k=0}^n a_k x^{n-k},\qquad q(x)=\sum_{k=0}^n b_k x^{n-k},
\]
define $p\boxplus_n q(x)=\sum_{k=0}^n c_k x^{n-k}$ by
\[
c_k=\sum_{i+j=k}\frac{(n-i)!(n-j)!}{n!(n-k)!}\,a_i b_j\qquad (0\le k\le n).
\]
For $p(x)=\prod_{i=1}^n (x-\lambda_i)$, set
\[
\Phi_n(p):=\sum_{i=1}^n\Bigl(\sum_{j\ne i}\frac{1}{\lambda_i-\lambda_j}\Bigr)^2,
\qquad\text{and }\ \Phi_n(p):=\infty\ \text{if $p$ has a multiple root.}
\]
Consider the inequality
\begin{equation}\label{eq:target}
\frac{1}{\Phi_n(p\boxplus_n q)}\ \ge\ \frac{1}{\Phi_n(p)}+\frac{1}{\Phi_n(q)}
\end{equation}
for monic real-rooted degree-$n$ polynomials $p,q$ (interpreting $1/\infty=0$).

\begin{enumerate}
\item[(i)] \emph{($n=2$) The inequality \eqref{eq:target} holds with equality for all monic real-rooted quadratics.}
\item[(ii)] \emph{(A strict example at $n=3$) There exist monic real-rooted cubics $p,q$ for which \eqref{eq:target} holds strictly.}
\item[(iii)] \emph{(Hermite family) For the probabilists' Hermite polynomials $H_n$ and their natural scalings, \eqref{eq:target} holds with equality for every $n$.}
\item[(iv)] \emph{(General $n$) The full inequality \eqref{eq:target} is not established by the arguments in the candidate solution; in particular, the appeal to a general ``finite free Stam inequality'' for this specific functional $\Phi_n$ is not substantiated by the cited literature.  The status of \eqref{eq:target} for general $n$ remains open based on the material presented here.}
\end{enumerate}
\end{theorem}

\begin{proof}
\medskip\noindent\textbf{A. Identification with symmetric additive convolution.}
The operation $\boxplus_n$ is the same (up to the standard sign convention for coefficients) as the \emph{symmetric additive convolution} $+_n$ introduced by Marcus--Spielman--Srivastava; see \cite[Definition~1.1]{MSS} for the defining coefficient formula (written there with the alternating-sign coefficient convention).  In particular, if $p$ and $q$ are real-rooted, then so is $p\boxplus_n q$ by \cite[Theorem~1.3]{MSS}.  (This real-rootedness preservation is conceptually relevant but not needed for the explicit computations in parts (i)–(iii) below.)

\medskip\noindent\textbf{B. A convenient identity for $\Phi_n$.}
Assume $p(x)=\prod_{i=1}^n (x-\lambda_i)$ has simple roots.  Then for each $i$,
\begin{equation}\label{eq:logderiv}
\frac{p''(\lambda_i)}{p'(\lambda_i)} \;=\; 2\sum_{j\ne i}\frac{1}{\lambda_i-\lambda_j}.
\end{equation}
Indeed, writing $p'(x)=\sum_i\prod_{j\ne i}(x-\lambda_j)$ and evaluating at $x=\lambda_i$ gives
$p'(\lambda_i)=\prod_{j\ne i}(\lambda_i-\lambda_j)$, while differentiating again and evaluating at
$x=\lambda_i$ yields
$p''(\lambda_i)=2\sum_{j\ne i}\prod_{k\ne i,j}(\lambda_i-\lambda_k)
=2p'(\lambda_i)\sum_{j\ne i}\frac{1}{\lambda_i-\lambda_j}$.
Thus
\begin{equation}\label{eq:Phi-logderiv}
\Phi_n(p)=\frac14\sum_{i=1}^n\Bigl(\frac{p''(\lambda_i)}{p'(\lambda_i)}\Bigr)^2.
\end{equation}
If $p$ has a multiple root then at least one denominator in \eqref{eq:logderiv} vanishes and $\Phi_n(p)=\infty$ by definition.

\medskip\noindent\textbf{C. Proof of (i): the case $n=2$ (equality).}
Let $p(x)=(x-\lambda_1)(x-\lambda_2)=x^2+a_1x+a_2$.  Then
\[
\Phi_2(p)=\Bigl(\frac{1}{\lambda_1-\lambda_2}\Bigr)^2+\Bigl(\frac{1}{\lambda_2-\lambda_1}\Bigr)^2
=\frac{2}{(\lambda_1-\lambda_2)^2}.
\]
Writing $\Delta(p):=(\lambda_1-\lambda_2)^2=a_1^2-4a_2$ (the discriminant), we have $\Phi_2(p)=2/\Delta(p)$ and
\begin{equation}\label{eq:Phi2disc}
\frac{1}{\Phi_2(p)}=\frac{\Delta(p)}{2}.
\end{equation}
Now let $q(x)=x^2+b_1x+b_2$ and set $r:=p\boxplus_2 q=x^2+c_1x+c_2$.  By the defining formula,
\[
c_1=a_1+b_1,\qquad
c_2=a_2+\frac12 a_1b_1+b_2.
\]
Hence
\[
\Delta(r)=c_1^2-4c_2
=(a_1+b_1)^2-4\Bigl(a_2+\frac12a_1b_1+b_2\Bigr)
=(a_1^2-4a_2)+(b_1^2-4b_2)=\Delta(p)+\Delta(q).
\]
Combining with \eqref{eq:Phi2disc} gives
\[
\frac{1}{\Phi_2(p\boxplus_2 q)}=\frac{\Delta(r)}{2}=\frac{\Delta(p)}{2}+\frac{\Delta(q)}{2}
=\frac{1}{\Phi_2(p)}+\frac{1}{\Phi_2(q)}.
\]
Thus \eqref{eq:target} holds with equality for $n=2$.

\medskip\noindent\textbf{D. A closed form for $\Phi_3$ and a strict example (ii).}
Let $p(x)=\prod_{i=1}^3(x-\lambda_i)=x^3+Ax^2+Bx+C$ have three distinct real roots and discriminant
\[
\Delta(p):=\prod_{1\le i<j\le 3}(\lambda_i-\lambda_j)^2.
\]
Set $s_1:=\lambda_1+\lambda_2+\lambda_3=-A$ and $s_2:=\lambda_1\lambda_2+\lambda_1\lambda_3+\lambda_2\lambda_3=B$.
A direct algebraic simplification starting from
\[
\Phi_3(p)=\sum_{i=1}^3\Bigl(\frac{1}{\lambda_i-\lambda_j}+\frac{1}{\lambda_i-\lambda_k}\Bigr)^2
\quad (\{i,j,k\}=\{1,2,3\})
\]
shows
\begin{equation}\label{eq:Phi3disc}
\Phi_3(p)=\frac{2(s_1^2-3s_2)^2}{\Delta(p)}=\frac{2(A^2-3B)^2}{\Delta(p)}.
\end{equation}
(One convenient route is to rewrite each term over the common denominator $\Delta(p)$ and then express the symmetric numerator in the basis
$\{s_1^4,s_1^2s_2,s_2^2,s_1s_3\}$, noting that the $s_1s_3$ coefficient cancels.)

Now take
\[
p(x)=x^3-3x+1\quad(A=0,\ B=-3),\qquad q(x)=x^3-3x\quad(A=0,\ B=-3).
\]
Their discriminants are those of depressed cubics:
\[
\Delta(x^3+Bx+C)=-4B^3-27C^2,
\]
so $\Delta(p)= -4(-3)^3-27\cdot 1^2=81$ and $\Delta(q)=-4(-3)^3-27\cdot 0^2=108$.
By \eqref{eq:Phi3disc},
\[
\Phi_3(p)=\frac{2(0-3(-3))^2}{81}=2,\qquad
\Phi_3(q)=\frac{2(0-3(-3))^2}{108}=\frac32.
\]
Next compute $r:=p\boxplus_3 q$.  Using the defining coefficient formula with $n=3$ and noting $a_1=b_1=0$,
\[
r(x)=x^3 + 0\cdot x^2 + (a_2+b_2)\,x + (a_3+b_3)
= x^3-6x+1,
\]
so $A=0$, $B=-6$, $C=1$, and $\Delta(r)=-4(-6)^3-27\cdot 1^2=837$.
Again by \eqref{eq:Phi3disc},
\[
\Phi_3(r)=\frac{2(0-3(-6))^2}{837}=\frac{648}{837}=\frac{24}{31}.
\]
Therefore
\[
\frac{1}{\Phi_3(p\boxplus_3 q)}=\frac{31}{24}
\quad\text{while}\quad
\frac{1}{\Phi_3(p)}+\frac{1}{\Phi_3(q)}=\frac12+\frac23=\frac76=\frac{28}{24},
\]
so \eqref{eq:target} holds strictly in this example.

\medskip\noindent\textbf{E. Equality for the Hermite family (iii).}
Let $H_n$ denote the probabilists' Hermite polynomial, characterized by the differential equation
\begin{equation}\label{eq:HermiteDE}
H_n''(x)-xH_n'(x)+nH_n(x)=0,
\end{equation}
and normalized to be monic of degree $n$ (equivalently, $H_n(x)=e^{-D^2/2}x^n$ as in \cite[\S1.1]{MSS}).
If $\lambda$ is a root of $H_n$, then $H_n(\lambda)=0$ and \eqref{eq:HermiteDE} gives
\[
\frac{H_n''(\lambda)}{H_n'(\lambda)}=\lambda.
\]
Combining with \eqref{eq:logderiv} yields the ``electrostatic'' identity
\[
\sum_{j\ne i}\frac{1}{\lambda_i-\lambda_j}=\frac{\lambda_i}{2}\qquad(\lambda_i\ \text{a root of }H_n),
\]
and hence
\begin{equation}\label{eq:PhiHermite}
\Phi_n(H_n)=\sum_{i=1}^n\Bigl(\frac{\lambda_i}{2}\Bigr)^2=\frac14\sum_{i=1}^n\lambda_i^2.
\end{equation}
For monic $H_n$, the coefficient of $x^{n-2}$ is $-\binom{n}{2}$, so $\sum_i\lambda_i^2 = n(n-1)$ (Newton sums), and
\begin{equation}\label{eq:PhiHermiteClosed}
\Phi_n(H_n)=\frac{n(n-1)}{4}.
\end{equation}

For $a>0$, define the scaling $H_{n,a}(x):=a^n H_n(x/a)$ (still monic).  Its roots are $a\lambda_i$, and from the scaling in the score
$\sum_{j\ne i}1/(a\lambda_i-a\lambda_j)=(1/a)\sum_{j\ne i}1/(\lambda_i-\lambda_j)$ we get
\[
\Phi_n(H_{n,a})=\frac{n(n-1)}{4a^2}
\quad\Longrightarrow\quad
\frac{1}{\Phi_n(H_{n,a})}=\frac{4a^2}{n(n-1)}.
\]
Marcus--Spielman--Srivastava note (via \cite[Theorem~1.2]{MSS} or directly from the coefficient formula \cite[Definition~1.1]{MSS}) that
\begin{equation}\label{eq:HermiteConvolution}
H_{n,a}\boxplus_n H_{n,b}=H_{n,c}\qquad\text{where }c=\sqrt{a^2+b^2}
\end{equation}
(see the displayed identity in \cite[\S1.1]{MSS} for Hermite polynomials).
Consequently,
\[
\frac{1}{\Phi_n(H_{n,a}\boxplus_n H_{n,b})}
=\frac{1}{\Phi_n(H_{n,c})}
=\frac{4(a^2+b^2)}{n(n-1)}
=\frac{1}{\Phi_n(H_{n,a})}+\frac{1}{\Phi_n(H_{n,b})},
\]
so \eqref{eq:target} holds with equality on this family.

\medskip\noindent\textbf{F. On the general inequality (iv).}
The candidate solution asserts that \eqref{eq:target} is a known general theorem (a ``finite free Stam inequality'') for the functional $\Phi_n$ above.
However, the standard (infinite-dimensional) free Stam inequality is a statement about Voiculescu's free Fisher information $\Phi^\ast$ under \emph{free additive convolution} (for freely independent noncommutative random variables); see \cite[\S3, $\Phi^\ast.4$]{Voi01}.
In contrast, a general Stam-type inequality for the \emph{finite-degree} functional $\Phi_n$ defined here and the finite convolution $\boxplus_n$ is not derived by the computations in parts (i)–(iii), and it does not follow from the cited finite-free convolution theory alone.
Related finite-degree entropy-power conjectures for roots/discriminants are discussed by Gribinski \cite[Conjecture~1]{Gri19}, who explicitly distinguishes such conjectures from a (future) development of a Fisher-information analogue.
Thus, based on the presented material, the general validity of \eqref{eq:target} for all $n$ remains open.
\end{proof}

\begin{thebibliography}{99}

\bibitem{MSS}
A.~Marcus, D.~Spielman, and N.~Srivastava,
\emph{Finite free convolutions of polynomials},
arXiv:1504.00350v2.

\bibitem{Gri19}
E.~Gribinski,
\emph{A notion of entropy on the roots of polynomials},
arXiv:1907.12826v1.

\bibitem{Voi01}
D.~Voiculescu,
\emph{Free entropy},
arXiv:math/0103168v1.

\end{thebibliography}

\end{result}


\end{document}